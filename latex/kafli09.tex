\lecture[22]{22. Runur og raðir}{lecture-text}
\date{12.~nóvember 2012}


\begin{document}

\subsection
	\maketitle


\section*{}
\subsection[t]{Runur}
 \subsubsection{22.1 Skilgreining}
{\em Runa} er raðaður listi af tölum. \pause 

Runa hefur fyrsta stak en
ekkert síðasta stak.  Stökin í runu eru oft númeruð með náttúrlegu
tölunum $1, 2, 3, \ldots$. Stökin eru þá
$$
a_1, a_2, a_3, a_4, a_5, \ldots
$$
\pause
 Runur eru táknaðar með 
$\{a_n\}_{n\in {\N}}$,
$\{a_n\}_{n=1}^\infty$ eða bara  $\{a_n\}$.
 


\subsection[t]{Takmarkaðar runur}
 \subsubsection{22.2 Skilgreining}
  Runa  $\{a_n\}$ er sögð {\em takmörkuð að neðan} ef til er
  tala $m$ þannig að 
  $$
  m\leq a_n
  $$ 
  fyrir allar náttúrlegar tölur $n$.  
  \pause

  Runan  er sögð {\em takmörkuð að ofan} ef til er
  tala $M$ þannig að 
  $$
  a_n\leq M
  $$ fyrir allar náttúrlegar tölur $n$. 
  \pause
  
  Runa sem er bæði takmörkuð að ofan og neðan er sögð {\em
  takmörkuð}.
 


\subsection[t]{Vöxtur runa}
 \subsubsection{22.3 Skilgreining}  
Runa  $\{a_n\}$ er sögð
\pause
\begin{itemize}
\item {\em vaxandi} ef $a_n\leq a_{n+1}$ fyrir öll $n$,\pause
\item {\em stranglega vaxandi} ef $a_n< a_{n+1}$ fyrir öll $n$,\pause
\item {\em minnkandi} ef $a_n\geq a_{n+1}$ fyrir öll $n$,\pause
\item {\em stranglega minnkandi} ef $a_n> a_{n+1}$ fyrir öll $n$.\pause
\end{itemize}


Runa kallast {\em einhalla} ef hún er annaðhvort vaxandi eða
minnkandi.
 


\subsection[t]{Víxlmerkjarunur}
 \subsubsection{22.4 Skilgreining}  
{\em Víxlmerkjaruna} er runa þannig að formerki skiptast
  á, annaðhvort
  $+, -, +, -, \ldots$ eða  $-, +, -, +, \ldots$. 
  \pause
  
  Einnig má lýsa
  þessu þannig að runa $\{a_n\}$ sé víxlmerkjaruna ef $a_na_{n+1}<0$
  fyrir öll $n$.
 


\subsection[t]{Samleitni}
 \subsubsection{22.5 Skilgreining}
Segjum að $\{a_n\}$ sé {\em samleitin} að tölu $L$ (eða {\em stefni á}
$L$) ef fyrir sérhverja tölu $\epsilon>0$ má finna náttúrlega tölu $N$
þannig að ef $n\geq N$ þá er 
$$|a_n-L|<\epsilon.$$
\pause

Ritað
$\lim_{n\rightarrow \infty}a_n=L$ og talan $L$ kallast 
\emph{markgildi rununnar}.   \pause

Sagt er að runa sé {\em samleitin}
ef $\lim_{n\rightarrow \infty}a_n$ er skilgreint, \pause
en annars er runan sögð
{\em ósamleitin}. 


\subsection[t]{Tengsl runa og falla}
 \subsubsection{22.6 Setning} 
Látum $f$ vera fall skilgreint á $\R$ og látum $\{a_n\}$ vera runu
þannig að $a_n=f(n)$ fyrir öll $n$.   Ef $\lim_{x\rightarrow
  \infty}f(x)=L$ þá er $\lim_{n\rightarrow\infty}a_n=L$.
 


\subsection[t]{Takmarkaðar runur og samleitni}
 \subsubsection{22.7 Setning}  
Látum $\{a_n\}$ vera runu.  Eftirfarandi tvö skilyrði eru jafngild:

\begin{itemize}
\item[(i)]  $\lim_{n\rightarrow\infty}a_n=L$, \pause
\item[(ii)] fyrir sérhvert $\epsilon>0$ eru aðeins endanlega margir
  liðir rununnar $\{a_n\}$ utan við bilið $(L-\epsilon, L+\epsilon)$.
 \end{itemize}
 

\pause

 \subsubsection{22.8 Fylgisetning} 
Samleitin runa er takmörkuð. 


\subsection[t]{Reiknireglur}
 \subsubsection{22.9 Setning}
Gerum ráð fyrir að 
runurnar $\{a_n\}$ og $\{b_n\}$ séu samleitnar.  Þá gildir:
\begin{itemize}
\item[(i)]  $\lim_{n\rightarrow\infty}(a_n\pm b_n)=
\lim_{n\rightarrow\infty}a_n\pm\lim_{n\rightarrow\infty}b_n$,\pause
\item[(ii)]  $\lim_{n\rightarrow\infty}ca_n=
c\lim_{n\rightarrow\infty}a_n$, þar sem $c$ er fasti,\pause
\item[(iii)]  $\lim_{n\rightarrow\infty}(a_n b_n)=
(\lim_{n\rightarrow\infty}a_n)(\lim_{n\rightarrow\infty}b_n)$,\pause
\item[(iv)] ef $\lim_{n\rightarrow\infty}b_n\neq 0$ þá er
$\lim_{n\rightarrow\infty}\frac{a_n}{b_n}=
\frac{\lim_{n\rightarrow\infty}a_n}{\lim_{n\rightarrow\infty}b_n}$,\pause
\item[(v)] ef $a_n\leq b_n$ fyrir öll $n$ sem eru nógu stór, þá er 
$$\lim_{n\rightarrow\infty}a_n\leq\lim_{n\rightarrow\infty}b_n,$$

(frasinn {\em fyrir öll $n$ sem eru nógu stór} þýðir að til er 
einhver tala $N$ þannig að skilyrðið gildir fyrir öll $n\geq N$),\pause
\item[(vi)]  (Klemmuregla)
ef $a_n\leq c_n\leq b_n$ fyrir öll $n$ sem eru nógu stór og 
$\lim_{n\rightarrow\infty}a_n=L=\lim_{n\rightarrow\infty}b_n$ þá er
runan $\{c_n\}$ samleitin og $$\lim_{n\rightarrow\infty}c_n=L.$$
 \end{itemize}
 


\subsection[t]{Samleitni einhalla runa}
 \subsubsection{22.8 Setning} 
Takmörkuð einhalla (vaxandi eða minnkandi) runa er samleitin. 




\end{document}
\lecture[23]{23. Raðir og samleitnipróf fyrir raðir}{lecture-text}
\date{14.~nóvember 2012}


\begin{document}

\subsection
	\maketitle


\section*{Raðír}
\subsection[t]{Raðir}
 \subsubsection{23.1 Skilgreining}
Látum $a_1, a_2, \ldots$ vera gefnar tölur. \pause
{\em Röðin}
$$\sum_{n=1}^\infty a_n \pause = a_1+a_2+a_3+\cdots$$
er skilgreind sem formleg summa liðanna $a_1, a_2, a_3, \ldots$.
 


\subsection[t]{Samleitni raða}
 \subsubsection{23.2 Skilgreining} 
Fáum í hendurnar röð  $\sum_{n=1}^\infty
a_n$ þar sem $a_1, a_2, \ldots$ eru tölur.  \pause
Skilgreinum 
$$s_n=a_1+a_2+\cdots+a_n$$
sem summa fyrstu $n$ liða raðarinnar. \pause 
Segjum að röðin 
$\sum_{n=1}^\infty a_n$ sé \emph{samleitin með summu} $s$ ef 
$$\lim_{n\rightarrow\infty}s_n=s.$$ \pause
Það er að segja, röðin er samleitin með summu $s$ ef
$$\lim_{n\rightarrow \infty}(a_1+a_2+\cdots+a_n)=s.$$ \pause
Ritum þá $$\sum_{n=1}^\infty a_n=s.$$
 


\subsection[t]{}
 \subsubsection{23.3 Setning}
Ef $A=\sum_{n=1}^\infty a_n$ og $B=\sum_{n=1}^\infty b_n$, þ.e.~báðar
raðirnar eru samleitnar, \pause þá gildir að

\begin{itemize} 
\item [(i)]  ef $c$ er fasti þá er $\sum_{n=1}^\infty ca_n=cA$, \pause 
\item [(ii)]  $\sum_{n=1}^\infty (a_n\pm b_n)=A\pm B$, \pause
\item [(iii)]  ef $a_n\leq b_n$ fyrir öll $n$ þá er $A\leq B$.
\end{itemize}
 
%

\pause

%\subsection[t]{}
 \subsubsection{23.4 Setning} 
Ef röð  $\sum_{n=1}^\infty a_n$ er samleitin þá er 
$$\lim_{n\rightarrow\infty}a_n=0.$$
 

\pause

\subsubsection{23.5 Athugasemd}
 Ef $\lim_{n \to \infty} a_n = 0$ þá  ekki víst að 
röðin $\sum_{n=1}^\infty a_n$ sé samleitin.




\section*{Samleitnipróf fyrir raðir}

\subsection[t]{Samleitnipróf I}
 \subsubsection{23.6 Setning}
Ef $\lim_{n\rightarrow\infty}a_n$ er ekki til eða 
$\lim_{n\rightarrow\infty}a_n\neq 0$ \pause þá er röðin 
$\sum_{n=1}^\infty a_n$ ekki samleitin.
 
%

%\subsection[t]{}
 \subsubsection{23.7 Setning} 
Gerum ráð fyrir að $a_n\geq 0$ fyrir allar náttúrlegar tölur $n$. \pause
Röðin $\sum_{n=1}^\infty a_n$ er þá annaðhvort samleitin eða 
ósamleitin að $\infty$ (þ.e.a.s.~hlutsummurnar 
$s_n=a_1+\cdots+a_n$ stefna á
$\infty$ þegar $n$ stefnir á $\infty$.)
 


\subsection[t]{Samleitnipróf II -- Samanburðarpróf}
 \subsubsection{23.8 Setning} 
Gerum ráð fyrir að $0\leq a_n\leq b_n$ fyrir allar náttúrlegar tölur
$n$.  \pause

\begin{itemize}
\item[(i)]  Ef $\sum_{n=1}^\infty b_n$ er samleitin þá er 
$\sum_{n=1}^\infty a_n$ líka samleitin. \pause
\item[(ii)]  Ef $\sum_{n=1}^\infty a_n$ er ósamleitin þá er 
$\sum_{n=1}^\infty b_n$ líka ósamleitin.
 \end{itemize}



\subsection[t]{Samleitnipróf III -- Heildispróf}
 \subsubsection{23.9 Setning}
Látum $f$ vera jákvætt, samfellt og minnkandi fall sem er skilgreint á
bilinu $[1, \infty)$. \pause  Fyrir sérhverja náttúrlega tölu $n$ setjum við
$a_n=f(n)$. \pause Þá eru röðin $\sum_{n=1}^\infty a_n$ og óeiginlega
heildið $\int_1^\infty f(x)\,dx$ annaðhvort bæði samleitin eða bæði
ósamleitin. 
 

 \subsubsection{23.10 Fylgisetning}
Röðin $\sum_{n=1}^\infty\frac{1}{n^{p}}$ er samleitin ef $p>1$ en 
ósamleitin ef
$p\leq 1$.
 


\subsection[t]{Samleitnipróf IV -- Markgildissamanburðarpróf}
 \subsubsection{23.11 Setning}
Gerum ráð fyrir að $a_n\geq 0$ og $b_n\geq 0$ fyrir allar náttúrlegar
tölur $n$ og $\lim_{n\rightarrow\infty}\frac{a_n}{b_n}=L$, þar sem
$L$ er tala eða $\infty$. \pause

\begin{itemize}
\item[(i)] Ef $L<\infty$ og  röðin 
$\sum_{n=1}^\infty b_n$ er samleitin þá er 
röðin $\sum_{n=1}^\infty a_n$ líka samleitin. \pause
\item[(ii)] Ef $L>0$ og  röðin 
$\sum_{n=1}^\infty b_n$ er ósamleitin þá er 
röðin $\sum_{n=1}^\infty a_n$ líka ósamleitin.
\end{itemize}
 



\end{document}
\lecture[24]{24. Samleitnipróf -- Alsamleitni}{lecture-text}
\date{19.~nóvember 2012}


\begin{document}

\subsection
	\maketitle


\section*{Samleitnipróf}
\subsection[t]{Samleitnipróf V -- Kvótapróf}
 \subsubsection{24.1 Setning}
Gerum ráð fyrir að $a_n>0$ fyrir öll $n$ og að markgildið 
$\rho=\lim_{n\rightarrow\infty}\frac{a_{n+1}}{a_n}$ sé skilgreint eða
að sé $\infty$.\pause
\begin{itemize}
\item[(i)]  Ef $0\leq\rho<1$ þá er röðin $\sum_{n=1}^\infty a_n$
  samleitin.\pause
\item[(ii)]  Ef $1<\rho\leq \infty$ þá er röðin $\sum_{n=1}^\infty a_n$
  ósamleitin.\pause
\item[(iii)]  Ef $\rho=1$ þá er ekkert hægt að fullyrða 
um hvort röðin $\sum_{n=1}^\infty a_n$
er samleitin eða  ósamleitin, hvor tveggja kemur til greina og nota
þarf aðrar aðferðir til að skera úr um það.
\end{itemize}
 


\subsection[t]{Samleitnipróf VI -- Rótarpróf}
 \subsubsection{24.2 Setning}
Gerum ráð fyrir að $a_n>0$ fyrir öll $n$ og að markgildið 
$\sigma=\lim_{n\rightarrow\infty}\sqrt[n]{a_n}$ sé skilgreint eða
að það sé $\infty$.\pause
\begin{itemize}
\item[(i)]  Ef $0\leq\sigma<1$ þá er röðin $\sum_{n=1}^\infty a_n$
  samleitin.\pause
\item[(ii)]  Ef $1<\sigma\leq \infty$ þá er röðin $\sum_{n=1}^\infty a_n$
  ósamleitin.\pause
\item[(iii)]  Ef $\sigma=1$ þá er ekkert hægt að fullyrða 
um hvort röðin $\sum_{n=1}^\infty a_n$
er samleitin eða  ósamleitin, hvor tveggja kemur til greina og nota
þarf aðrar aðferðir til að skera úr um það.
\end{itemize}
 


\section*{Alsamleitni}

\subsection[t]{Alsamleitni}
 \subsubsection{24.3 Skilgreining}
Röð $\sum_{n=1}^\infty a_n$ er sögð vera {\em alsamleitin} ef röðin 
$\sum_{n=1}^\infty |a_n|$ er samleitin.
 

\pause

 \subsubsection{24.4 Setning}
Röð sem er alsamleitin er samleitin.  
 


\subsection[t]{Skilyrt samleitni}
 \subsubsection{24.5 Athugasemd}
Til eru samleitnar raðir, t.d.\
röðin  $\sum_{n=1}^\infty \frac{(-1)^{n-1}}{n}$, sem eru ekki
alsamleitnar. 
 

\pause

 \subsubsection{24.6 Skilgreining}
Samleitin röð $\sum_{n=1}^\infty a_n$ er sögð vera {\em skilyrt
  samleitin} 
ef röðin $\sum_{n=1}^\infty |a_n|$ er ósamleitin.
 


\subsection[t]{Samleitnipróf VII -- Víxlmerkjaraðapróf}
 \subsubsection{24.7 Setning}
Gerum ráð fyrir að 
\begin{itemize}
\item[(i)]  $a_n\geq 0$ fyrir öll $n$ (frekar jákvæðir liðir),\pause
\item[(ii)] $a_{n+1}\leq a_n$ fyrir öll $n$ (frekar minnkandi),\pause
\item[(iii)]  $\lim_{n\rightarrow\infty} a_n=0$ (stefnir á 0).\pause
\end{itemize}
Þá er víxlmerkjaröðin 
$$\sum_{n=1}^\infty (-1)^{n-1}a_n=a_1-a_2+a_3-a_4+\cdots$$
samleitin.
 

 \subsubsection{24.8 Fylgisetning}
Gerum ráð fyrir að runa $\{a_n\}$ uppfylli skilyrðin sem gefin eru í
24.7. \pause

Látum $s_n$ tákna summu $n$ fyrstu liða raðarinnar 
$\sum_{n=1}^\infty (-1)^{n-1}a_n$ og táknum summu raðarinnar með $s$.
Þá gildir að $|s-s_n|\leq |a_{n+1}|$.
 


\subsection[t]{Umröðun}
 \subsubsection{24.9 Setning}
Dæmi um umröðun á liðum raðar $\sum_{n=1}^\infty a_n$ er
$$a_{10}+a_9+\cdots+a_1+a_{100}+a_{99}+\cdots+a_{11}+
a_{1000}+a_{999}+\cdots.$$\pause
\begin{itemize}
\item[(i)]  Ef röðin $\sum_{n=1}^\infty a_n$ er alsamleitin þá skiptir
  engu máli hvernig liðum raðarinnar er umraðað, summan verður
  alltaf sú sama.  \pause
\item[(ii)]  Ef röðin $\sum_{n=1}^\infty a_n$ er skilyrt samleitin
og $L$ einhver rauntala, eða $\pm\infty$ þá er hægt að
umraða liðum raðarinnar þannig að summan eftir umröðun verði
$L$.
\end{itemize}
 

\pause
\subsubsection{Með öðrum orðum}
 Liðum skilyrt samleitinnar raðar má umraða þannig að summan getur orðið hvað sem
er, \pause það skiptir því máli í hvaða röð við leggjum saman.




\subsection[t]{Kvótaröð}
 \subsubsection{24.10 Setning}
  Röðin 
$$
\sum_{n=0}^\infty a^n
$$ 
kallast \emph{kvótaröð}. \pause Hún er samleitin ef $-1<a<1$ og þá er 
$$
\sum_{n=0}^\infty a^n = \frac{1}{1-a}.
$$ 
 



\subsection[t]{Kíkisröð}
 \subsubsection{24.11 Setning}
Röðin 
$$
\sum_{n=1}^\infty \frac{1}{n(n-1)}
$$
kallast \emph{kíkisröð}. \pause Hún er samleitin og
$$
\sum_{n=1}^\infty \frac{1}{n(n-1)} =1.
$$

 



\end{document}
\lecture[25]{25. Veldaraðir}{lecture-text}
\date{21.~nóvember 2012}


\begin{document}

\subsection
	\maketitle


\section*{Veldaraðir}
\subsection[t]{Veldaraðir}
 \subsubsection{25.1 Skilgreining}  
Röð á forminu 
$$\sum_{n=0}^\infty a_n(x-c)^n=a_0+a_1(x-c)+a_2(x-c)^2+\cdots$$
kallast {\em veldaröð} með {\em miðju} í punktinum $c$.



\subsection[t]{Samleitnibil}
 \subsubsection{25.2 Setning.} 
Um sérhverja veldaröð
$\sum_{n=0}^\infty a_n(x-c)^n$ gildir eitt af þrennu:\pause
\begin{itemize}
\item[(i)]   Röðin er aðeins samleitin fyrir $x=c$.\pause
\item[(ii)]  Til er jákvæð tala $R$ þannig að veldaröðin er alsamleitin
  fyrir öll $x$ þannig að $|x-c|<R$ og ósamleitin fyrir öll $x$
  þannig að $|x-c|>R$. \pause Mögulegt að röðin sé samleitin í öðrum eða
  báðum punktunum $x=c-R$ og $x=c+R$, en það er líka mögulegt að röðin
  sé ósamleitin í þeim báðum.\pause
\item[(iii)]  Röðin er samleitin fyrir allar rauntölur $x$.
\end{itemize}



\subsection[t]{Miðja og samleitnigeisli}
 \subsubsection{25.3 Skilgreining og útlistun}
Látum   $\sum_{n=0}^\infty a_n(x-c)^n$ vera veldaröð.\pause
\begin{itemize}
\item[(a)]  Talan $c$ kallast {\em miðja} eða {\em samleitnimiðja}
  veldaraðarinnar. \pause
\item[(b)]  Í tilviki (ii) í Setningu 25.2 er röðin alsamleitin á
  bilinu $(c-R, c+R)$ og ósamleitin fyrir utan bilið $[c-R, c+R]$.\pause
  
Mögulegt er að röðin sé samleitin (alsamleitin eða skilyrt
  samleitin)  í öðrum eða báðum punktunum $x=c-R$ og $x=c+R$ (þarf að
  athuga sérstaklega).  \pause

Talan $R$ er kölluð {\em samleitnigeisli}
  raðarinnar.  \pause

Í tilfelli 25.2(i) þegar röðin er bara samleitin fyrir
  $x=c$ setjum við $R=0$ \pause og í tilfelli 25.2(iii) þegar röðin er
  samleitin fyrir allar rauntölur $x$ þá setjum við $R=\infty$.\pause
\item[(c)]  {\em Samleitnibil} veldaraðarinnar $\sum_{n=0}^\infty
  a_n(x-c)^n$ er mengi allra gilda $x$ þannig að röðin er samleitin. \pause
  Setning 25.2 sýnir að þetta mengi er alltaf bil.
\end{itemize}



\subsection[t]{Samleitnibil}
\subsubsection{25.4 Athugasemd}
\begin{itemize}
\item Í tilfelli 25.2(i) er samleitnibilið $\{c\}$.\pause
\item Í tilfelli 25.2(ii) eru fjórir möguleikar eftir því hvort röðin er
samleitin í hvorugum, öðrum eða báðum punktunum $x=c-R$ og $x=c+R$.  
Samleitnibilið getur verið \pause
\begin{itemize}
 \item[$\bullet$] $(c-R, c+R)$,\pause
\item[$\bullet$] $[c-R, c+R)$,\pause
\item[$\bullet$]  $(c-R, c+R]$, \pause 
\item[$\bullet$] $[c-R, c+R]$.\pause
\end{itemize}
\item Í tilfelli 25.2(iii) er samleitnibilið $(-\infty, \infty)$.
\end{itemize}




\subsection[t]{Samleitnipróf}
 \subsubsection{25.4 Setning}  
Látum $\sum_{n=0}^\infty a_n(x-c)^n$ vera veldaröð.\pause
\begin{itemize}
\item[(i)]  \emph{Kvótapróf:}  Gerum ráð fyrir að 
$L=\lim_{n\rightarrow\infty}\left|\frac{a_{n+1}}{a_n}\right|$ sé til
eða $\infty$. \pause 

Þá hefur veldaröðin  $\sum_{n=0}^\infty a_n(x-c)^n$
samleitnigeisla 
$$R=\pause \left\{\begin{array}{ll}
\infty & \mbox{ef }L=0,\pause\\
\frac{1}{L} & \mbox{ef }0<L<\infty,\pause\\
0 & \mbox{ef }L=\infty.\\
\end{array}
\right.
$$\pause
\item[(ii)]  \emph{Rótarpróf:} \pause Gerum ráð fyrir að 
$L=\lim_{n\rightarrow\infty}\sqrt[n]{|a_n|}$ sé til
eða $\infty$. \pause 
Þá hefur veldaröðin  $\sum_{n=0}^\infty a_n(x-c)^n$
samleitnigeisla 
$$R=\pause \left\{\begin{array}{ll}
\infty & \mbox{ef }L=0,\pause\\
\frac{1}{L} & \mbox{ef }0<L<\infty,\pause\\
0 & \mbox{ef }L=\infty.\\
\end{array}
\right.
$$
\end{itemize}


\subsection[t]{Setning Abels}
 \subsubsection{25.5 Setning}
Fallið $f$ skilgreint á samleitnibili með 
$$
f(x)=\sum_{n=0}^\infty a_n(x-c)^n
$$ \pause 
er samfellt á {\bf öllu}
samleitnibili veldaraðarinnar.  

Ef samleitnigeislinn er $0<R<\infty$
og röðin er samleitin í punktinum $x=c+R$ \pause
þá er 
$$\lim_{x\rightarrow (c+R)^-}f(x)=f(c+R)=\sum_{n=0}^\infty
a_n((c+R)-c)^n=\sum_{n=0}^\infty a_nR^n.$$

\pause

Eins ef röðin er samleitin í punktinum $x=c-R$ þá er
$$\lim_{x\rightarrow (c-R)^+}f(x)=f(c-R)=\sum_{n=0}^\infty
a_n((c-R)-c)^n=\sum_{n=0}^\infty a_n(-R)^n.$$
 



\end{document}
\lecture[26]{26. Veldaraðir -- Taylorraðir}{lecture-text}
\date{23.~nóvember 2011}


\begin{document}

\subsection
	\maketitle


\section*{}
\subsection[t]{Diffrað lið fyrir lið}
 \subsubsection{26.1 Setning}
Látum 
$\sum_{n=0}^\infty a_n(x-c)^n=a_0+a_1(x-c)+a_2(x-c)^2+a_3(x-c)^3+\cdots$
vera veldaröð með miðju í $c$ og samleitnigeisla $R$. \pause

Fyrir $x\in(c-R, c+R)$ skilgreinum við 
$$
f(x)=\sum_{n=0}^\infty a_n(x-c)^n.$$
\pause

Fallið $f$ er diffranlegt og 
$$f'(x)=\sum_{n=1}^\infty
na_n(x-c)^{n-1}=a_1+2a_2(x-c)+3a_3(x-c)^2+\cdots$$
\pause og röðin fyrir $f'(x)$ er samleitin fyrir öll
$x\in(c-R, c+R)$.
\pause

Þetta þýðir að við getum diffrað veldaraðir lið fyrir lið.
 


\subsection[t]{Samfelldni}
Þar sem diffranleg föll eru samfelld þá fæst eftirfarandi.
\pause
 \subsubsection{26.2 Fylgisetning}
   Fallið $f$ er samfellt á $(c-R, c+R)$.
 


\subsection[t]{Heildað lið fyrir lið}
 \subsubsection{26.3 Setning}
Látum 
$\sum_{n=0}^\infty a_n(x-c)^n=a_0+a_1(x-c)+a_2(x-c)^2+a_3(x-c)^3+\cdots$
vera veldaröð með miðju í $c$ og samleitnigeisla $R$. \pause
\pause

Fyrir $x\in(c-R, c+R)$ skilgreinum við 
$f(x)=\sum_{n=0}^\infty a_n(x-c)^n$.
\pause

Fallið $f$ hefur stofnfall
\begin{multline*}
F(x)=\sum_{n=0}^\infty \frac{a_n}{n+1}(x-c)^{n+1} \\
=a_0(x-c)+\frac{a_1}{2}(x-c)^2+\frac{a_2}{3}(x-c)^3+
\frac{a_3}{4}(x-c)^4+\cdots
\end{multline*}
\pause og röðin fyrir $F(x)$  er samleitin fyrir öll
$x\in(c-R, c+R)$.
\pause

Þetta þýðir að við getum heildað veldaraðir lið fyrir lið.
 





\subsection[t]{Tengsl $f$ við $a_n$}
 \subsubsection{26.4 Setning}
Látum 
$\sum_{n=0}^\infty a_n(x-c)^n=a_0+a_1(x-c)+a_2(x-c)^2+a_3(x-c)^3+\cdots$
vera veldaröð með miðju í $c$ og samleitnigeisla $R$. \pause

 Fyrir
$x\in(c-R, c+R)$ skilgreinum við 
$$f(x)=\sum_{n=0}^\infty a_n(x-c)^n.$$
\pause

Fallið $f$ er $k$-sinnum diffranlegt fyrir $k=1, 2, 3, \ldots$ og
$$a_k=\frac{f^{(k)}(c)}{k!}.$$
 


\subsection[t]{Fáguð föll}
 \subsubsection{26.5 Skilgreining}  
Fall $f$ þannig að til er veldaröð $\sum_{n=0}^\infty a_n(x-c)^n$ með
samleitnigeisla $R>0$ þannig að 
$$f(x)=\sum_{n=0}^\infty a_n(x-c)^n$$
fyrir öll $x\in(c-R, c+R)$ kallast {\em fágað} (raunfágað) í punktinum
$c$.
 

\pause

\subsubsection{26.6 Athugasemd}
 Dæmi um raunfáguð föll eru margliður, ræð föll, hornaföll, veldisföll
og lograr.





\subsection[t]{Taylorröð}
 \subsubsection{26.7 Skilgreining}
Gerum ráð fyrir að fall $f(x)$ sé óendanlega oft diffranlegt í
punktinum $x=c$, \pause (þ.e.a.s.~$f^{(k)}(c)$ er til fyrir $k=0, 1, 2,
\ldots$).  

\pause

Veldaröðin 
\begin{align*}
\sum_{n=0}^\infty \frac{f^{(n)}(c)}{n!}(x-c)^n = & f(c)+f'(c)(x-c)+
\frac{f''(c)}{2}(x-c)^2 \\ & + \frac{f'''(c)}{3!}(x-c)^3 
 + \frac{f^{(iv)}(c)}{4!}(x-c)^4 + \cdots 
\end{align*}
kallast {\em Taylorröð} með miðju í $x=c$ fyrir $f(x)$. \pause 

Ef svo vill til að $c=0$ þá er oft talað um {\em Maclaurinröð}.



\subsection[t]{Upprifjun}
 \subsubsection{26.8 Skilgreining og setning}   Taylormargliða með miðju í $c$ fyrir $f$ er
skilgreind sem margliðan
\begin{align*}
	P_n(x)& =\sum_{n=0}^n \frac{f^{(k)}(c)}{n!}(x-c)^n \\
	&=f(c)+f'(c)(x-c)+ \frac{f''(c)}{2}(x-c)^2+\cdots+\frac{f^{(n)}(c)}{n!}(x-c)^n.
\end{align*}
\pause 
Skekkjan í $n$-ta stigs Taylornálgun er
$R_n(x)=f(x)-P_n(x)$. \pause 

Til er tala $X$ sem liggur á milli $c$ og $x$
þannig að 
$$
R_n(x)=\frac{f^{(n+1)}(X)}{(n+1)!}(x-c)^{n+1}.
$$



\subsection[t]{Taylorröð og skekkjan}
 \subsubsection{26.9 Setning.}  Gerum ráð fyrir að $f$ sé fall sem er óendanlega
oft diffranlegt í punktinum $c$. 
\pause

Fyrir fast gildi á $x$ þá er Taylorröðin 
$$
\sum_{n=0}^\infty
\frac{f^{(n)}(c)}{n!}(x-c)^n
$$ samleitin með summu $f(x)$ \pause ef og aðeins
ef $$\lim_{n\rightarrow\infty}R_n(x)=0$$



\subsection[t]{Tvíliðuröðin}
 \subsubsection{26.10 Setning}
Fyrir $x$ þannig að $|x|<1$ og rauntölu $r$ gildir að \pause
\begin{align*}
(1+x)^r =& 1+rx+\frac{r(r-1)}{2!}x^2+
\frac{r(r-1)(r-2)}{3!}x^3 \\ 
&+\frac{r(r-1)(r-2)(r-3)}{4!}x^4+\cdots\\
=& 1+ \sum_{n=1}^\infty \frac{r(r-1)(r-2)\cdots(r-n+1)}{n!}x^n.
\end{align*}
\pause
\subsubsection{26.11 Athugasemd}
 Ef $r \in \N$ þá gefur summan að ofan einfaldlega stuðlanna þegar búið er að 
margfalda upp úr svigum, og summan er því endanleg því þegar $n \geq r+1$ þá
verða stuðlarnir 0.
\pause

Ef hins vegar $r\notin \N$ þá er enginn stuðlanna 0.





\subsection[t]{Taylorraðir nokkra falla með miðju $c=0$}
 \subsubsection{26.12 Setning}{\small
\begin{align*}
e^x&=\sum_{n=0}^\infty\frac{x^n}{n!}
    =1+x+\frac{x^2}{2}+\frac{x^3}{3!}
    +\cdots
  &\mbox{fyrir öll }x\\
\sin x&=  \sum_{n=0}^\infty\frac{(-1)^n}{(2n+1)!}x^{2n+1}
    =x-\frac{x^3}{3!}+\frac{x^5}{5!}-\frac{x^7}{7!}+\cdots
     &\mbox{fyrir öll }x\\ 
\cos x&=  \sum_{n=0}^\infty\frac{(-1)^n}{(2n)!}x^{2n}
    =1-\frac{x^2}{2!}+\frac{x^4}{4!}-\frac{x^6}{6!}+\cdots
    &\mbox{fyrir öll }x\\
\frac{1}{1-x}&=\sum_{n=0}^\infty x^n
    =1+x+x^2+x^3+\cdots
&\mbox{fyrir }-1<x<1\\
\frac{1}{(1-x)^2}&=\sum_{n=1}^\infty nx^{n-1}
    =1+2x+3x^2+4x^3+\cdots
&\mbox{fyrir }-1<x<1\\
\end{align*}}



\subsection[t]{Fleiri Taylorraðir}
\subsubsection{26.12 Setning, framhald}{\small
\begin{align*}
\ln(1+x)&=  \sum_{n=1}^\infty\frac{(-1)^{n-1}}{n}x^n
    =x-\frac{x^2}{2}+\frac{x^3}{3}-\frac{x^4}{4}+\cdots
    &\mbox{fyrir }-1<x\leq 1\\
\tan^{-1} x&=  \sum_{n=0}^\infty\frac{(-1)^n}{2n+1}x^{2n+1}
    =x-\frac{x^3}{3}+\frac{x^5}{5}-\frac{x^7}{7}+\cdots
    &\mbox{fyrir }-1\leq x\leq 1\\\\
\sinh x&=  \sum_{n=0}^\infty\frac{x^{2n+1}}{(2n+1)!}
    =x+\frac{x^3}{3!}+\frac{x^5}{5!}+\frac{x^7}{7!}+\cdots
    &\mbox{fyrir öll } x\\
\cosh x&=  \sum_{n=0}^\infty\frac{x^{2n}}{(2n)!}
    =1+\frac{x^2}{2!}+\frac{x^4}{4!}+\frac{x^6}{6!}+\cdots
    &\mbox{fyrir öll } x\\
\end{align*} }
 


\end{document}
\lecture[27]{27. Taylorraðir}{lecture-text}
\date{28.~nóvember 2011}


\begin{document}

\subsection
	\maketitle


\section*{}
\subsection[t]{Taylorröð}
 \subsubsection{27.1 Skilgreining}
Gerum ráð fyrir að fall $f(x)$ sé óendanlega oft diffranlegt í
punktinum $x=c$, \pause þ.e.a.s.~$f^{(k)}(c)$ er til fyrir $k=0, 1, 2,
\ldots$).  

\pause

Veldaröðin 
\begin{align*}
\sum_{n=0}^\infty \frac{f^{(n)}(c)}{n!}(x-c)^n &=f(c)+f'(c)(x-c)+
\frac{f''(c)}{2}(x-c)^2+\frac{f'''(c)}{3!}(x-c)^3 \\
& + \frac{f^{(iv)}(c)}{4!}(x-c)^4 + \cdots 
\end{align*}
kallast {\em Taylorröð} með miðju í $x=c$ fyrir $f(x)$. \pause 

Ef svo vill til að $c=0$ þá er oft talað um {\em Maclaurinröð}.



\subsection[t]{Upprifjun}
 \subsubsection{27.2 Skilgreining og setning}   Taylormargliða með miðju í $c$ fyrir $f$ er
skilgreind sem margliðan
\begin{align*}
	P_n(x)& =\sum_{n=0}^n \frac{f^{(k)}(c)}{n!}(x-c)^n \\
	&=f(c)+f'(c)(x-c)+ \frac{f''(c)}{2}(x-c)^2+\cdots+\frac{f^{(n)}(c)}{n!}(x-c)^n.
\end{align*}
\pause 
Skekkjan í $n$-ta stigs Taylornálgun er
$R_n(x)=f(x)-P_n(x)$. \pause 

Til er tala $X$ sem liggur á milli $c$ og $x$
þannig að 
$$
R_n(x)=\frac{f^{(n+1)}(X)}{(n+1)!}(x-c)^{n+1}.
$$



\subsection[t]{Taylorröð og skekkjan}
 \subsubsection{27.3 Setning.}  Gerum ráð fyrir að $f$ sé fall sem er óendanlega
oft diffranlegt í punktinum $c$. 
\pause

Fyrir fast gildi á $x$ þá er Taylorröðin 
$$
\sum_{n=0}^\infty
\frac{f^{(n)}(c)}{n!}(x-c)^n
$$ samleitin með summu $f(x)$ \pause ef og aðeins
ef $$\lim_{n\rightarrow\infty}R_n(x)=0$$



\subsection[t]{Tvíliðuröðin}
 \subsubsection{27.4 Setning}
Fyrir $x$ þannig að $|x|<1$ og rauntölu $r$ gildir að \pause
\begin{align*}
(1+x)^r =& 1+rx+\frac{r(r-1)}{2!}x^2+
\frac{r(r-1)(r-2)}{3!}x^3 \\ 
&+\frac{r(r-1)(r-2)(r-3)}{4!}x^4+\cdots\\
=& 1+ \sum_{n=1}^\infty \frac{r(r-1)(r-2)\cdots(r-n+1)}{n!}x^n.
\end{align*}
\pause
\subsubsection{27.5 Athugasemd}
 Ef $r \in \N$ þá gefur summan að ofan einfaldlega stuðlanna þegar búið er að 
margfalda upp úr svigum, og summan er því endanleg því þegar $n \geq r+1$ þá
verða stuðlarnir 0.
\pause

Ef hins vegar $r\notin \N$ þá er enginn stuðlanna 0.





\subsection[t]{Taylorraðir nokkra falla með miðju $c=0$}
 \subsubsection{27.6 Setning}{\small
\begin{align*}
e^x&=\sum_{n=0}^\infty\frac{x^n}{n!}
    =1+x+\frac{x^2}{2}+\frac{x^3}{3!}
    +\cdots
  &\mbox{fyrir öll }x\\
\sin x&=  \sum_{n=0}^\infty\frac{(-1)^n}{(2n+1)!}x^{2n+1}
    =x-\frac{x^3}{3!}+\frac{x^5}{5!}-\frac{x^7}{7!}+\cdots
     &\mbox{fyrir öll }x\\ 
\cos x&=  \sum_{n=0}^\infty\frac{(-1)^n}{(2n)!}x^{2n}
    =1-\frac{x^2}{2!}+\frac{x^4}{4!}-\frac{x^6}{6!}+\cdots
    &\mbox{fyrir öll }x\\
\frac{1}{1-x}&=\sum_{n=0}^\infty x^n
    =1+x+x^2+x^3+\cdots
&\mbox{fyrir }-1<x<1\\
\frac{1}{(1-x)^2}&=\sum_{n=1}^\infty nx^{n-1}
    =1+2x+3x^2+4x^3+\cdots
&\mbox{fyrir }-1<x<1\\
\end{align*}}



\subsection[t]{Fleiri Taylorraðir}
\subsubsection{27.6 Setning, framhald}{\small
\begin{align*}
\ln(1+x)&=  \sum_{n=1}^\infty\frac{(-1)^{n-1}}{n}x^n
    =x-\frac{x^2}{2}+\frac{x^3}{3}-\frac{x^4}{4}+\cdots
    &\mbox{fyrir }-1<x\leq 1\\
\tan^{-1} x&=  \sum_{n=0}^\infty\frac{(-1)^n}{2n+1}x^{2n+1}
    =x-\frac{x^3}{3}+\frac{x^5}{5}-\frac{x^7}{7}+\cdots
    &\mbox{fyrir }-1\leq x\leq 1\\\\
\sinh x&=  \sum_{n=0}^\infty\frac{x^{2n+1}}{(2n+1)!}
    =x+\frac{x^3}{3!}+\frac{x^5}{5!}+\frac{x^7}{7!}+\cdots
    &\mbox{fyrir öll } x\\
\cosh x&=  \sum_{n=0}^\infty\frac{x^{2n}}{(2n)!}
    =1+\frac{x^2}{2!}+\frac{x^4}{4!}+\frac{x^6}{6!}+\cdots
    &\mbox{fyrir öll } x\\
\end{align*} }
 




\end{document}
