\documentclass[icelandic,a4paper,12pt]{article}
\usepackage{beamerarticle}

\input{beamer.tex}\lecture[1]{2. Markgildi og samfelldni}{lecture-text}
\date{29. ágúst 2015}

\newcommand{\C}{{\mathbb  C}}
\newcommand{\Z}{{\mathbb Z}}
\newcommand{\R}{{\mathbb  R}}
\newcommand{\N}{{\mathbb  N}}
\newcommand{\Q}{{\mathbb Q}}
\newcommand{\Sin}{{\text{Sin}}}
\newcommand{\Tan}{{\text{Tan}}}
\newcommand{\Cos}{{\text{Cos}}}
\newcommand{\Cosh}{{\text{Cosh}}}
\newcommand{\arsinh}{{\text{arsinh}}}
\newcommand{\arcosh}{{\text{arcosh}}}
\newcommand{\artanh}{{\text{artanh}}}


\begin{document}
\setcounter{tocdepth}{2}
\tableofcontents

\section{Diffurjöfnur}
\subsection{Diffurjöfnur}
\subsubsection{Skilgreining: Diffurjafna}  
Ritum $y=y(x)$ sem fall af $x$. 

\emph{Diffurjafna} er jafna á forminu 
$$F(x, y, y', y'', \ldots, y^{(n)})=0$$
þar sem $F$ er fall (formúla) í $n+1$ breytistærð.  

Jafnan er sögð vera af $n$-ta \emph{stigi} ef hæsta afleiða $y$ sem kemur fyrir í
formúlu er $n$.

.. note::
   Deildajafna, afleiðujafna og diffurjafna eru samheiti yfir sama hlutinn. 

\subsubsection{Dæmi}
Það að finna stofnfall fyrir fall $f$ er jafngilt því að leysa 
fyrsta stigs diffurjöfnuna 
$$
y'(x) = f(x),
$$
eða með framsetningunni úr 20.1,
$$
F(x,y') = f(x) - y'(x) = 0.
$$

%\subsection{Aðgreinanlegar diffurjöfnur}
\subsubsection{Skilgreining: Aðgreinanleg diffurjafna} 
Fyrsta stigs diffurjafna sem má rita á forminu 
$$\frac{dy}{dx}=f(x)g(y)$$ 
kallast \emph{aðgreinanleg} (e. seperable).  
Það er, þátta má hægri hliðina  þannig að annar þátturinn
er bara fall af $x$ og hinn þátturinn er bara fall af $y$.

Umritum jöfnuna yfir á formið $$\frac{dy}{g(y)}=f(x)\,dx.$$ 
\textbf{Athugið}: Ekkert $x$ í vinstri hlið, ekkert $y$ í hægri hlið. 

Síðan smellum við heildum á báðar hliðar og fáum að 
$$\int\frac{dy}{g(y)}=\int f(x)\,dx.$$
 
Reiknum stofnföllin hægra og vinstra megin í jöfnunni
$$\int\frac{dy}{g(y)}=\int f(x)\,dx.$$
og munum eftir að setja inn heildunarfasta (einn er nóg).  
Þá höfum við jöfnu sem tengir saman $x$ og $y$, og inniheldur engar afleiður af $y$. 
Út frá þeirri jöfnu má fá upplýsingar um eiginleika lausnarinnar $y$.  
Stundum er hægt að einangra $y$ og fá þannig formúlu fyrir lausn diffurjöfnunar.

.. todo::
  Dæmi

%\subsection{Línulegar diffurjöfnur}
\subsubsection{Skilgreining: Línuleg diffurjafna} 
Diffurjafna á forminu
$$a_n(x)y^{(n)}+a_{n-1}(x)y^{(n-1)}+\cdots+a_1(x)y'+a_0(x)y=f(x)$$
kallast \emph{línuleg diffurjafna}. 
Hún er $n$-ta stigs ef $a_n(x)$ er
ekki fastafallið $0$.  

Ef $f$ er fastafallið $0$ þá er jafnan sögð
\emph{óhliðruð} (e. homogeneous)  en ef $f$ er ekki fastafallið $0$ þá
er hún sögð \emph{hliðruð} (e. nonhomogeneous). 

\subsection{Línulegar fyrsta stigs diffurjöfnur}
\subsubsection{Línulegar fyrsta stigs diffurjöfnur}
Almenna línulega fyrsta stigs jöfnu má rita á forminu
$$y'+p(x)y=q(x).$$

Samsvarandi óhliðruð jafna er $$y'+p(x)y=0.$$

Skilgreinum $\mu(x)=\int p(x)\,dx$ (eitthvert stofnfall).  Þá er 
$$y(x)=e^{-\mu(x)}\int e^{\mu(x)}q(x)\,dx$$
lausn á diffurjöfnunni.  

\subsubsection{Athugasemd}
Þegar þið reiknið $\mu(x)=\int p(x)\,dx$ þá
megið þið sleppa heildunarfastanum, en \textbf{ekki} þegar þið reiknið heildið 
$\int e^{\mu(x)}q(x)\,dx$.


\subsection{Línulegar annars stigs diffurjafnur með fastastuðla}
\subsubsection{Skilgreining}  
\emph{Línuleg annars stigs diffurjafna með
fastastuðla} er diffurjafna á forminu 
$$ay''+by'+cy=f(x)$$
þar sem $a, b$ og $c$ eru fastar. 

Jafnan er sögð \emph{óhliðruð} (e. homogeneous) ef fallið $f(x)$ er fastafallið 0. 

\subsubsection{Skilgreining: Kennijafna} 
Jafnan $ar^2+br+c=0$ kallast \emph{kennijafna} (e. auxiliary equation) diffurjöfnunnar $ay''+by'+cy=0$.

\subsubsection{Setning}  
Ef föllin $y_1(x)$ og $y_2(x)$ eru lausnir á diffurjöfnunni $ay''+by'+cy=0$ þá er fallið
$$y(x)=Ay_1(x)+By_2(x),$$ þar sem $A$ og $B$ eru fastar, líka lausn. 

Ef $y_2(x)$ er ekki fastamargfeldi af $y_1(x)$ þá má skrifa \textbf{sérhverja} lausn $y(x)$ á
diffurjöfnunni $ay''+by'+cy=0$ á forminu $$y(x)=Ay_1(x)+By_2(x),$$ þar sem $A$ og $B$ eru fastar.

\subsubsection{Setning}
\begin{description}
\item[Tilvik I] \emph{Kennijafnan $ar^2+br+c=0$ hefur tvær ólíkar rauntölulausnir
$r_1$ og $r_2$.}

Fallið $$y(x)=Ae^{r_1x}+Be^{r_2x}$$ er alltaf lausn sama hvernig fastarnir $A$ og $B$ eru valdir og
sérhverja lausn má rita á þessu formi.

\item[Tilvik II] \emph{Kennijafnan $ar^2+br+c=0$ hefur bara eina rauntölulausn
$k=-\frac{b}{2a}$.}

Fallið $$y(x)=Ae^{kx}+Bxe^{kx}$$ er alltaf lausn sama hvernig fastarnir $A$ og $B$ 
eru valdir og sérhverja lausn má rita á þessu formi.

\item[Tilvik III] \emph{Kennijafnan $ar^2+br+c=0$ hefur engar rauntölulausnir.}

Setjum $k=-\frac{b}{2a}$ og $\omega=\frac{\sqrt{4ac-b^2}}{2a}$. 

Rætur kennijöfnunnar eru $r_1=k+i\omega$ og $r_2=k-i\omega$.

Fallið
$$y(x)=Ae^{kx}\cos(\omega x)+Be^{kx}\sin(\omega x)$$
er alltaf lausn sama hvernig fastarnir $A$ og $B$ eru valdir  og
sérhverja lausn má rita á þessu formi.
\end{description}

\subsubsection{Setning}
Látum $y_{\rm p}(x)$ vera einhverja lausn á hliðruðu jöfnunni
$$
  ay''+by'+cy=f(x).
$$ 

Látum $y_1(x)$ og $y_2(x)$ vera lausnir sem fást úr 21.4 á óhliðruðu jöfnunni
$$
ay''+by'+cy=0.
$$

.. todo::
  laga tilvísun

Sama hvernig fastarnir $A$ og $B$ eru valdir þá er fallið 
$$y(x)=Ay_1(x)+By_2(x)+y_{\rm p}(x)$$ 
alltaf lausn á diffurjöfnunni  $ay''+by'+cy=f(x)$ og sérhverja lausn
má skrifa á þessu formi.

\subsection{Ágiskanir}
\subsubsection{Ágiskanir} 
(Sjá ramma í grein 17.6)

.. todo::
  tilvísun í bók?

Leysa á jöfnu $ay''+by'+cy=f(x)$.



Látum $P_n(x)$ standa fyrir einhverja $n$-ta stigs margliðu og látum
$A_n(x)$ og $B_n(x)$ tákna $n$-ta stigs margliður með óákveðnum stuðlum. 

\begin{itemize}
\item Ef $f(x)=P_n(x)$ þá giskað á $y_{\rm p}(x)=x^mA_n(x)$.
\item Ef $f(x)=P_n(x)e^{rx}$ þá giskað á $y_{\rm p}(x)=x^mA_n(x)e^{rx}$.
\item Ef $f(x)=P_n(x)e^{rx}\sin(kx)$ 
þá giskað á $y_{\rm p}(x)=x^me^{rx}[A_n(x)\cos(kx)+B_n(x)\sin(kx)]$.
\item Ef $f(x)=P_n(x)e^{rx}\cos(kx)$ 
þá giskað á $y_{\rm p}(x)=x^me^{rx}[A_n(x)\cos(kx)+B_n(x)\sin(kx)]$.
\end{itemize} 
Hér táknar $m$ minnstu töluna af tölunum 0, 1, 2 sem tryggir að enginn
liður í ágiskuninni sé lausn á óhliðruðu jöfnunni $ay''+by'+cy=0$.


\subsection{Samantekt}
\subsubsection{Aðskiljanlegar jöfnur}
Jöfnur sem hægt er að rita á forminu
$$
\frac{dy}{dx} = f(x)g(y),
$$
má leysa með því að heilda og einangra $y$ út úr
$$
\int \frac 1{g(y)}\, dy = \int f(x)\, dx.
$$
 
\subsubsection{Línulegar fyrsta stigs jöfnur}
Lausn við jöfnu á forminu 
$$
  y'(x) + p(x)y = q(x)
$$
er gefin með 
$$
y(x) = e^{-\mu(x)} \int e^{\mu(x)} q(x)\, dx,
$$
þar sem $\mu(x) = \int p(x)\, dx$.

.. todo::
  Annars stigs jöfnur og ágiskanir.

\end{document}
